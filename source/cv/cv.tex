\documentclass[a4paper, 10pt]{article}
\usepackage{geometry}
\usepackage{hyperref}
\usepackage{enumitem}
\usepackage{ctex}  % 添加中文支持
\usepackage{xcolor}  % 添加颜色支持
\definecolor{tsinghua_purple}{HTML}{660874}  % 定义颜色
\usepackage{titlesec}

\geometry{left=1in,right=1in,top=1in,bottom=1in}

\titleformat{\section}{\color{tsinghua_purple}\zihao{4}\heiti}{}{0em}{}

\title{个人简历}
\date{}

\begin{document}

% Reduce space around title
\maketitle
\vspace{-2em}

\section*{个人简介}
我叫董业恺, 曾任清华大学计算机系科协网络部副主席. 我拥有丰富的软件工程, 网络技术与网络安全实践经验, 对网络安全和下一代互联网, 天地一体化互联网等方向有浓厚的兴趣.

\section*{教育背景}
\noindent
\begin{tabular}{p{0.17\textwidth}p{0.83\textwidth}}
    2022.9 - 2023.8 & 清华大学电机工程与应用电子技术系 \newline
        \fontsize{8pt}{10pt}\selectfont 本科二年级转系到清华大学计算机科学与技术系 \\

    2023.9 - 至今 & 清华大学计算机科学与技术系 \newline
        \fontsize{8pt}{10pt}\selectfont 本科在读,主修计算机科学与技术专业 \\
\end{tabular}

\section*{技术技能}
\begin{itemize}[left=0pt]
    \item \textbf{网络协议分析与应用}: 我熟练掌握包括但不限于 SNMP, STP, DHCP, ARP, TCP, IPMI 等协议的原理和应用方法, 能准确定位网络故障, 快速解决网络问题;
    \item \textbf{网络配置与架构}: 我具备大量配置交换机, 路由器等网络设施的经验, 能快速调通中等规模的网络并利用策略路由等手段完成网络配置.
    \item \textbf{网络探测与分析}: 我能够在短时间内综合利用工具绘制目标网络的骨干拓扑, 理清网络架构, 完成仿真模型, 分析网络脆弱性.
    \item \textbf{网络安全与维护}: 我擅长发现与利用网络安全漏洞, 在 THUCTF 等活动中取得了良好的成绩, 在校内有 `水木白帽' 称号, 协助信息化技术中心发现并处理了多处网络安全隐患.
    \item \textbf{软件工程与开发}: 我开发了多个软件工程项目, 部分项目已交付并投入使用. 我也积极参与社区项目的贡献, 对软件工程有读到的见解.
    \item \textbf{服务运维与管理}: 我接手维护了酒井 ID (\href{https://stu.cs.tsinghua.edu.cn}{stu.cs.tsinghua.edu.cn}), Git9 (\href{https://git.net9.org}{git.net9.org}), THUInfo APP (\href{https://app.cs.tsinghua.edu.cn}{app.cs.tsinghua.edu.cn}) 等多个项目, 接触了 Grafana 等自动化运维的工具与手段.
\end{itemize}

\section*{获得荣誉}
\begin{itemize}[left=0pt]
    \item 清华大学计算机科学与技术系优秀共青团员
    \item 清华大学优秀科协骨干
    \item 清华大学软件工程课程优秀大作业
    \item THUCTF 2022 二等奖, 新生特别奖
\end{itemize}

\section*{项目经历}
\noindent
\begin{tabular}{p{0.17\textwidth}p{0.83\textwidth}}
    2023.1 - 2023.5 & \textbf{RainDropWeb 雨实验平台开发 (个人项目)}
        \fontsize{8pt}{10pt}\selectfont
        \begin{itemize}[left=0pt,topsep=0pt,partopsep=0pt,parsep=0pt,itemsep=0pt]
            \item 为数字电路实验课程的实验平台开发了具有跨平台, 响应式特性的客户端.
            \item 逆向了其通信协议以支持自定义的实现;
        \end{itemize} \\

    2023.9 - 至今 & \textbf{THUInfo APP 维护 (个人项目)}
        \fontsize{8pt}{10pt}\selectfont
        \begin{itemize}[left=0pt,topsep=0pt,partopsep=0pt,parsep=0pt,itemsep=0pt]
            \item 维护了整合校园常用功能的 THUInfo APP 软件;
        \end{itemize} \\

    2024.2 - 2024.5 & \textbf{即时通讯系统开发 (课程大作业)}
        \fontsize{8pt}{10pt}\selectfont \begin{itemize}[left=0pt,topsep=0pt,partopsep=0pt,parsep=0pt,itemsep=0pt]
            \item 在软件工程课程上开发了即时通讯系统, 被评为优秀大作业;
        \end{itemize} \\

    2024.6 - 2024.11 & \textbf{清华大学学生社团管理系统 (科协项目)}
        \fontsize{8pt}{10pt}\selectfont \begin{itemize}[left=0pt,topsep=0pt,partopsep=0pt,parsep=0pt,itemsep=0pt]
            \item 带队完成了共青团清华大学委员会的学生社团管理系统项目, 部署在 \href{https://shetuan.student.tsinghua.edu.cn}{shetuan.student.tsinghua.edu.cn};
        \end{itemize} \\

    2023.9 - 2024.11 & \textbf{计算机系学生节网络架构和弹幕机重构 (科协项目)}
        \fontsize{8pt}{10pt}\selectfont \begin{itemize}[left=0pt,topsep=0pt,partopsep=0pt,parsep=0pt,itemsep=0pt]
            \item 两次组织了计算机系学生节的网络运维工作, 重构了现有网络架构, 给出了详细的文档;
            \item 主持了对弹幕机的重构, 扩展了系统功能, 简化了系统结构, 使之稳定运行;
        \end{itemize} \\

    2024.10 - 2024.12 & \textbf{南区地下二层机房改造工作 (社团项目)}
        \fontsize{8pt}{10pt}\selectfont \begin{itemize}[left=0pt,topsep=0pt,partopsep=0pt,parsep=0pt,itemsep=0pt]
            \item 参与设计了南区地下二层学生兴趣团队服务器机房, 重构了网络架构并部署了状态监测系统;
        \end{itemize} \\
    
    2025.2 - 2025.6 & \textbf{清华大学计算机系《软件工程》课程助教}
        \fontsize{8pt}{10pt}\selectfont \begin{itemize}[left=0pt,topsep=0pt,partopsep=0pt,parsep=0pt,itemsep=0pt]
            \item 建立了常用软件源本地缓存, 使得选课学生的作业 CI 时间大幅缩短;
            \item 协助维护了 SECoder 平台, 对同学们的大作业进行了细致的测试, 反馈了大量问题.
        \end{itemize}
\end{tabular}

\section*{社会工作}
\noindent
\begin{tabular}{p{0.17\textwidth}p{0.83\textwidth}}
    2023.5 - 2024.4 & \textbf{计算机系科协网络部干事}
        \fontsize{8pt}{10pt}\selectfont
        \begin{itemize}[left=0pt,topsep=0pt,partopsep=0pt,parsep=0pt,itemsep=0pt]
            \item 整理了科协存量服务文档, 得到了科协存量信息系统目录
        \end{itemize} \\

    2024.7 - 2024.8 & \textbf{计算机系科协与清华算协联合暑培讲师}
        \fontsize{8pt}{10pt}\selectfont
        \begin{itemize}[left=0pt,topsep=0pt,partopsep=0pt,parsep=0pt,itemsep=0pt]
            \item 讲授 Linux \& Git 和 Django 有关技术
        \end{itemize} \\

    2024.5 - 2025.5 & \textbf{计算机系科协网络部副主席}
        \fontsize{8pt}{10pt}\selectfont
        \begin{itemize}[left=0pt,topsep=0pt,partopsep=0pt,parsep=0pt,itemsep=0pt]
            \item 完成了科协服务向新系馆机房迁移
            \item 校内 Dify LLM 平台部署
            \item 完成了科协存量服务运维与清退
            \item 完成了 FTP9 和 Cloud9 部署等工作
        \end{itemize}
\end{tabular}

\section*{课程成绩}
\noindent 我取得满绩的课程包括但不限于:
\begin{itemize}[left=20pt]
    \setlength{\itemsep}{0pt}
    \item 计算机网络原理
    \item 计算机网络安全技术
    \item 无线移动网络技术
    \item 软件工程
    \item 计算机系统概论
\end{itemize}

\section*{联系方式}
\noindent 博客链接:\href{https://aajax.top}{aajax.top}; 邮箱:\href{mailto:i@aajax.top}{i@aajax.top}

\end{document}
